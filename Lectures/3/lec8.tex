\include{6001mac}
\begin{document}
\psetheader{IAP 2005}{Lecture 8}

\section*{Tags}

\begin{verbatim}
; professor abstraction
(define (make-professor name salary)
  (list name salary))

(define (professor-name prof)
  (first prof))

(define (professor-salary prof)
  (second prof))

; graduate student abstraction
(define (make-gradstudent name salary)
  (list name salary))

(define (gradstudent-name grad)
  (first grad))

(define (gradstudent-salary grad)
  (second grad))
\end{verbatim}

Given a list that contains both professors and graduate students,
compute the total cost of their salaries.

\begin{verbatim}
(define (total-cost people-list)
\end{verbatim}
\vspace{3in}

\section*{Association Lists}

\subsection*{Scheme}
\begin{enumerate}
\item {\tt assoc} - {\tt (assoc {\it key} {\it alist})} - returns
association containing matching key or \#f.
\vspace{.5in}
\item {\tt del-assoc} - {\tt (del-assoc {\it key} {\it alist})} -
returns a new alist with association with matching key removed.
\vspace{.5in}
\end{enumerate}

\subsection*{Problems}

\begin{enumerate}
\item Evaluate the following expressions, first guessing then checking
with Scheme.
\begin{verbatim}
(define alst (list (list 1 2) (list 3 4) (list 5 6)))

(assoc 4 alst)

(assoc 3 alst)

(assoc 5 (cons (list 5 12) alst))

(del-assoc 5 alst)

(define alst2 (list (list "foo" 17) (list "bar" 42) (list "baz" 54)))

(assoc "foo" alst2)

(del-assoc "bar" alst2)

(assoc "yummy" alst2)

(assoc "yummy" alst)
\end{verbatim}

\item Rewrite {\tt lookup} from homework 7 using {\tt assoc}.

\begin{verbatim}
(define (lookup word thesaurus)
\end{verbatim}
\vspace{1in}

\end{enumerate}


\section*{Trees}

\begin{verbatim}
(define (make-node val left right)
  (list "node" val left right))

(define (node? x)
  (and (pair? x) (string=? (car x) "node")))

(define (node-val node)
  (second node))
(define (node-left node)
  (third node))
(define (node-right node)
  (fourth node))

(define (leaf? x)
  (not (node? x)))
\end{verbatim}

\begin{enumerate}
\item Write {\tt tree-contains?}, which returns true if the tree contains the value as a leaf.
\begin{verbatim}
(define (tree-contains? tree val)
\end{verbatim}
\vspace{1in}

\item Write {\tt sum-tree}, which returns the sum of the leaves of the tree.
\begin{verbatim}
(define (sum-tree tree)
\end{verbatim}
\vspace{1in}

\end{enumerate}

\begin{verbatim}
(define (insert-list elem lst)
  (if (null? lst)
      (list elem)
      (if (< elem (car lst))
          (cons elem lst)
          (cons (car lst) (insert-list elem (cdr lst))))))

(define (avg v1 v2)
  (/ (+ v1 v2) 2))
\end{verbatim}

\begin{enumerate}
\setcounter{enumi}{2}
\item Complete {\tt insert-tree}, which returns a {\it new tree} with the
value added to the correct place in the tree.
\begin{verbatim}
(define (insert-tree elem tree)
  (if (leaf? tree)
      (if (= elem tree)
          INSERT1
          (if (< elem tree)
              INSERT2
              INSERT3))
      (if (< elem (node-val tree))
          (make-node (node-val tree)
                     INSERT4
                     (node-right tree))
          (make-node (node-val tree)
                     (node-left tree)
                     INSERT5))))
\end{verbatim}
\vspace{2in}

\end{enumerate}

\newpage

\section*{Animal Guessing Game}

Download {\tt lec8.scm} from the website.

\begin{enumerate}
\item Write the {\tt animal} abstraction

\item Write the {\tt ask-about-animal} procedure, which should take an
animal as input and ask the player if that is their animal
\begin{verbatim}
(ask-about-animal (make-animal "elephant"))

Is it a elephant (y or n)?         ;  ('n' key was struck)
;Value: #f
\end{verbatim}

\item Look at the {\tt play-game} procedure.  This procedure uses a
{\tt guesser} procedure combined with some {\tt knowledge} of animals
in order to guess the player's animal.  Let's start off by using a
list of {\tt animal}s as the knowledge.  Implement {\tt list-guesser},
which takes in a list of animals and asks the player about them until
it guesses the animal or runs out of knowledge.  If it succeeds, use
{\tt print-msg} to print out a victory message.  If it runs out of
knowledge without guessing the animal, print out {\tt "I give up."}.

\item Look more closely at the {\tt play-game} procedure.  It uses the
return value of the {\tt guesser} as the new knowledge to use when
playing the next game.  Thus, we want to have the guesser return the
knowledge.  The reason {\tt play-game} does this is it allows the
guesser to ask a couple more questions when it fails to extend its
knowledge to cover the situation where it lost:

\begin{verbatim}
(play-game new-list-guesser sample-list)

Is it a elephant (y or n)? n

Is it a hummingbird (y or n)? n
I give up.

What was your animal
(Please enter a string (surrounded by "s) and use C-x, C-e to submit it)
"thesaurus"

play again (y or n)? y

Is it a elephant (y or n)? n

Is it a hummingbird (y or n)? n

Is it a thesaurus (y or n)? y
Yay!

play again (y or n)? n
;Value: (("animal" "elephant") ("animal" "hummingbird") ("animal" "thesaurus"))
\end{verbatim}

Write a {\tt new-list-guesser} procedure which returns a new improve
knowledge list each time it runs.

\item Most games of guess an animal are not played by repeated asking
the player about every animal you know.  By asking other yes-no
questions, the scope of possible animals can be narrowed to a small
range.  The sounds like a job for trees!

Implement the {\tt question} abstraction: a question is a node in our
knowledge tree.

\item Implement the {\tt ask-question} procedure which asks the player
the question.

\item The leaves of the tree are animals.  Implement {\tt
tree-guesser} that takes in a tree as its knowledge and searchs the
tree, asking questions to decide whether the left or right branch is
the correct one.

\begin{verbatim}
(play-game new-tree-guesser sample-tree)

does it fly (y or n)? n

Is it a elephant (y or n)? y
Yay!

play again (y or n)? y

does it fly (y or n)? y

Is it a hummingbird (y or n)? y
Yay!

play again (y or n)? n
;Value: ("question" "does it fly" ("animal" "hummingbird") ("animal" "elephant"))
\end{verbatim}

\item Once again, we should write our guesser such that it improves
its knowledge each time.  The {\tt improve-tree} procedure has been
given to you.  Write {\tt new-tree-guesser}.


\end{enumerate}

\end{document}
