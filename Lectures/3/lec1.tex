\include{6001mac}
\begin{document}
\psetheader{IAP 2005}{Lecture 1}

\section*{Glossary}
\begin{itemize}
\item [program] collection of procedures and static data that
  accomplishes a specific task.
\item [procedure] a piece of code that when called with arguments
  computes and returns a result; possibly with some side-effects.  In
  scheme, procedures are normal values like numbers.
\item [function] see procedure; they're equivalent in scheme.  Some
  other languages make a distinction.
\item [argument] An input variable to a procedure.  A new version of
  the variable is created every time the procedure is called.
\item [parameter] see argument.
\item [expression] A single valid scheme statement.  {\tt 5}, {\tt (+
  3 4)}, and {\tt (if (lambda (x) x) 5 (+ 3 4))} are expressions.
\item [value] The result of a evaluating an expression. 5, 7, and 5
  respectively.
\item [type] Values are classified into types.  Some types: numbers,
  booleans, strings, lists, and procedures.  Generally, types are
  disjoint (any value falls into exactly one type class).
\item [call] Verb, the action of invoking, jumping to, or using a
  procedure.
\item [apply] Calling a procedure.  Often used as ``apply procedure p
  to arguments a1 and a2.''
\item [pass] Usage ``pass X to Y.''  When calling procedure Y, supply
  X as one of the arguments.
\item [side-effect] In relation to an expression or procedure, some
  change to the system that does not involve the expression's value.  
\item [iterate] To loop, or ``do'' the same code multiple times.
\item [variable] A name that refers to a exactly one value.
\item [binding] Also verb ``to bind''.  The pairing of a name with a
  value to make a variable.
\item [recurse] In a procedure, to call that same procedure again.
\end{itemize}

\newpage

\section*{Values}
\begin{enumerate}
\item {\bf Numbers}
\vspace{.5in}
\item {\bf Booleans}
\vspace{.5in}
\item {\bf Strings}
\vspace{.5in}
\item {\bf Procedures}
\vspace{.5in}
\item {\bf Lists}
\vspace{.5in}
\end{enumerate}

\section*{Scheme}
\begin{enumerate}
\item {\bf Basic Elements}
  \begin{enumerate}
  \item {\it self-evaluating} - expressions whose value is the same as
    the expression.
    \vspace{1in}
  \item {\it names} - Name is looked up in the symbol table to find
    the value associated with it.  Names may be made of any collection
    of characters that doesn't start with a number.
    \vspace{1in}
  \end{enumerate}
\item {\bf Combination}\\
  {\tt (} {\it procedure} {\it arguments-separated-by-spaces} {\tt )}\\
  Value is determined by evaluating the expression for the procedure
  and applying the resulting value to the value of the arguments.
  \vspace{1in}

\item {\bf Special Forms}
  \begin{enumerate}
  \item {\large {\it define}} - {\tt (define {\it name value})}\\ The
    name is bound to the result of evaluating the the value.  Return
    value is {\it unspecified}.
    \vspace{1in}
  \item {\large {\it if}} - {\tt (if {\it test consequent alternative})}\\ If
    the value of the test is \underline{not false (\#f)}, evaluate the
    consequent, otherwise evaluate the alternative.
    \vspace{1in}
  \end{enumerate}
\end{enumerate}

\section*{Problems}
Run scheme, open a new file ``class1.scm''.  All of your answers for
the following problems should end up in this file.

\begin{enumerate}
\item {\large {\it Evaluation}} - For each expression:
\begin{enumerate}
\item Write the type of the expression
\item Write your guess as to the expression's return value.
If the expression is erroneous simply indicate ``error'' for the value.  
If the expression returns an unspecified value, write whatever you want!
\item Evaluate the expression, and copy the response from the *scheme* buffer.
\end{enumerate}

\begin{verbatim}
4

5.5

4.2e1

(+ 1 2)

(7)

(* (+ 7 8) (   -  5 6))



(define one 1)

(define two (+ 1 one))

(define five 3)

(+ five two)

(define biggie-size *)

(define hamburger 1)

(biggie-size hamburger five)

(= 7 (+ 3 4))

(= #t #f)

((+ 5 6))

biggie-size

(if #t 1 (+ 3 0))

(if (if #f #t #f) #f #t)

(if (if (= hamburger five) 3 7) 
    (+ (if (= (+ 1 one) two)
           3
           5)
       7)
    "yay")

(/ 256
   (if (> five (+ two 1))
       (/ 7 (- hamburger 1))
       2))
\end{verbatim}

\item {\large {\it Five}} - write an expression that evaluates to 3.
\vspace{.5in}
\item {\LARGE {\it Five}} - write a {\it more interesting} expression
that evaluates to 3.
\vspace{.5in}
\newpage
\item {\large {\it Define X}} - for each of the following expressions:
\begin{enumerate}
\item Identify the variables are are {\it unbound}.
\item Supply definitions (ie {\tt (define x ...)}) for each of the
variables that make the expression evaluate to the target value.
\item Type in the expressions and verify that your solution gives the
correct result.
\end{enumerate}

\begin{verbatim}
(+ x (* y 3))
;Value: 13

(if q (if r 3 4) 7)
;Value: 4

(+ 7 (if z (+ a z) 3))
;Value: 11

(= yum (* -1 (+ yum 2)))
;Value: #t

(< (if (not thirty-four) 34 thirty-four)
   (+ thirty-four (* seventy-seven (if seventy-seven -1 1))))
;Value: #f
\end{verbatim}

\newpage
\item {\large {\it Primitive Procedures}} - for each of the following
expressions:
\begin{enumerate}
\item Identify the primitive procedures which you don't already know
\item Write down a guess as to what the primitive procedure does.
\item Look it up in the MITScheme reference manual (you may find the
index handy).
\item Write an example usage of the procedure and test it to see that
it works as you suspect it does.
\item Fill in the blanks in the original expression such that it
evaluates to the target value.
\end{enumerate}

\begin{verbatim}
(+ 3 (abs ________))
;Value: 5

(if (> (expt 2 ____) 15)
    (sqrt 2)
    (sqrt 4))
;Value: 1.4142135623730951

(string-append "foo" ___ "baz")
;Value: "foobarbaz"

(string-append
  (number->string
    (gcd 35 _____))
  " rules!")
;Value: "7 rules!"

(define s _______)
(if (string-suffix? "yowzah" s)
    (+ 7 (string-search-forward "ow" s))
    #f)
;Value: 14
\end{verbatim}

\end{enumerate}

\end{document}
