\include{6001mac}
\begin{document}
\psetheader{IAP 2005}{Lecture 6}

\section*{Scheme}

\begin{enumerate}
\item {\bf Special Forms}
  \begin{enumerate}
  \item {\large {\it and}} - {\tt (and {\it arg1 arg2 ...})}\\
    Evaluates arguments from left to right, stopping at the first one
    that evaluates to false and returning false.  Should all the
    arguments evaluate true-ishly, returns the value of the last
    argument.
    \vspace{1in}
  \item {\large {\it or}} - {\tt (or {\it arg1 arg2 ...})}\\ Evaluates
    arguments from left to right, stopping at the first one that
    evaluates to true-ish and returns that value.  Should all the
    arguments evaluate to false, returns false.
    \vspace{1in}
  \end{enumerate}
\end{enumerate}

\section*{Higher Order Procedures}

\begin{verbatim}
(define sum
  (lambda (f x y dx)
\end{verbatim}

\newpage

\section*{Types}

\section*{Problems}

For each expression, write the type of the {\bf value} that results
from evaluating the expression.  Ignore {\tt define} expressions.
\begin{verbatim}
4

(+ 1 1)

(lambda (x) (+ x 1))

(lambda (x) (= x 1))

(define square
  (lambda (x) (* x x)))

square

(square 5)

(define a
  (lambda (f) (+ (f 5) 1)))

a

(a square)

(define b
  (lambda (x y)
    (+ (a x) y)))

b

(b square 4)

(define c
  (lambda (x)
    (lambda (y)
      (+ x y))))

c

(c 5)
\end{verbatim}

\end{document}
